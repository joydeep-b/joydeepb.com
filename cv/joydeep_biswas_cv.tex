\documentclass[Times]{article}
\usepackage{times}
\usepackage[margin=1.25in]{geometry}
\usepackage{graphicx}
\usepackage{hyperref}
\usepackage{verbatim}
\usepackage{multirow}
\usepackage{color}
\usepackage{titlesec}
\usepackage[normalem]{ulem}
\usepackage{sectsty}


%%%%%%%%%%%%%%%%%%%%%%%%%%%%%%%%%%%%%%%%%
\usepackage[american]{babel}
\usepackage{csquotes}
\usepackage[style=numeric,sorting=ydnt,defernumbers=true,maxbibnames=99]{biblatex}

\DeclareFieldFormat{labelnumber}{\mkbibdesc{#1}}

\makeatletter

% Print labelnumber as actual number, plus item total, minus one
\newrobustcmd{\mkbibdesc}[1]{%
  \number\numexpr\csuse{bbx@itemtotal}+#1+0\relax}

% Initialize category counters
\def\bbx@initcategory#1{\csnumgdef{bbx@count@#1}{0}}
\forlistloop{\bbx@initcategory}{\blx@categories}

% Increment category counters
\def\bbx@countcategory#1{%
  \ifentrytype{#1}
    {\csnumgdef{bbx@count@#1}{\csuse{bbx@count@#1}+1}%
     \addtocategory{#1}{\thefield{entrykey}}%
     \listbreak}
    {}}
\AtDataInput{\forlistloop{\bbx@countcategory}{\blx@categories}}

% Modify \bibbycategory to set item total
\patchcmd{\blx@bibcategory}
  {\blx@key@heading{#1}}
  {\blx@key@heading{#1}%
   \csnumdef{blx@labelnumber@\the\c@refsection}{0}%
   \csnumgdef{bbx@itemtotal}{\csuse{bbx@count@#1}}}
  {}{}

\makeatother

\DeclareBibliographyCategory{inproceedings}
\DeclareBibliographyCategory{article}
\DeclareBibliographyCategory{misc}
\DeclareBibliographyCategory{thesis}

\defbibheading{bibliography}{\section*{Publications}}
\defbibheading{article}{\subsection*{Journal Articles}}
\defbibheading{thesis}{\subsection*{PhD Thesis}}
\defbibheading{inproceedings}{\subsection*{Conference Papers}}
\defbibheading{misc}{\subsection*{Other Publications}}

\addbibresource{references.bib}
%%%%%%%%%%%%%%%%%%%%%%%%%%%%%%%%%%%%%%%%%

\setlength{\parindent}{0pt}
\sectionfont{\sectionrule{0pt}{0pt}{-1ex}{1pt}}
\subsectionfont{\underline}


\newcommand{\funding}[1]{#1\\}
\renewcommand{\funding}[1]{\\}


\begin{document}

\title{
\vspace{-3em}
\textbf{Joydeep Biswas}\\
\vspace{0.5em}
\normalsize{
Computer Science Department, University of Texas at Austin, TX 78712, USA.\\
\href{mailto:joydeepb@cs.utexas.edu}{joydeepb@cs.utexas.edu}, \href{http://www.joydeepb.com}{http://www.joydeepb.com}\vspace{-3em}}
}
\date{}
\maketitle


% \begin{resume}
\section*{Current Appointment}

Associate Professor, Computer Science Department,
University of Texas at Austin\\
\vspace{-0.5em}\\
% Adjunct Associate Professor, College of Information and Computer Sciences,
% University of Massachusetts Amherst\\
% \vspace{-0.5em}\\
Technical Advisor, Consumer Robotics, Amazon Lab 126
% \section*{Interests}
% Mobile robot localization, mapping and navigation, 3D perception, multi-robot
% systems, robust and reliable robots.

\section*{Education}
\begin{tabular}{ p{1.2cm} l l }
  2014  & Ph.D. in Robotics  & Carnegie Mellon University \\
  2010  & M.S. in Robotics  & Carnegie Mellon University \\
  2008  & B.Tech. in Engineering Physics & Indian Institute of Technology, Bombay \\
\end{tabular}

\section*{Achievements and Awards}
\begin{tabular}{ p{1.2cm} l }
  2023  & JP Morgan Faculty Research Award \\
  2022  & 1st Place, Benchmark Autonomous Robot Navigation (BARN) Challenge,
          ICRA 2022 \\
  2021  & NSF CAREER Award\\
  2019  & Student Best Poster Award, Northrop Grumman University Symposium \\
  2019  & IJCAI Early Career Spotlight \\
  2019  & Amazon Research Award \\
  2018  & JP Morgan AI Faculty Research Award \\
  2018  & Best Demo Award, AAMAS 2018 \\
  2018  & 5th place, RoboCup 2018 Small Size League,
      \emph{UMass MinuteBots}, Faculty Team Leader\\
  2017  & Lower Bracket 1st place, RoboCup 2017 Small Size League, \\
      &\emph{UMass MinuteBots}, Faculty Team Leader\\
  2015  & Siebel Scholar, Class of 2015 \\
  2015  & 1st place, RoboCup 2015 Small Size League,
     \emph{CMDragons}, Student Team Leader\\
  2014  & 2nd place, RoboCup 2014 Small Size League,
     \emph{CMDragons}, Student Team Leader\\
  2013  & 2nd place, RoboCup 2013 Small Size League,
     \emph{CMDragons}, Student Team Leader\\
  2010  & 2nd place, RoboCup 2010 Small Size League,
     \emph{CMDragons}, Team Member\\
\end{tabular}

% \section*{Research Interests}

% {\em
% Robot Perception, Motion Planning, Control Systems, AI, Deployed Robot Systems
% }\\

%
% My ultimate goal is to have self-sufficient autonomous mobile robots working in human environments,
% performing tasks accurately and robustly. In support of this goal, I am interested in research in
% perception, planning, and control applied to autonomous mobile robots. My research in perception
% involves developing models and representations for a dynamic world, and algorithms to build and
% perform inference based on such models. My interests in planning include motion planning, multi-robot
% coordination, and task-based planning in domains including service mobile robots, and robot soccer.

%\newpage

\section*{Employment History}
\begin{tabular}{ p{2cm} l p{8cm}}
  2019 - Present & Associate Professor & Computer Science Department,
  University of Texas at Austin, TX, USA \\
  2019 - 2023 & Assistant Professor & Computer Science Department,
  University of Texas at Austin, TX, USA \\
  2019 - 2023 & Adjunct Assistant Professor &  College of Information and Computer Sciences, University of Massachusetts Amherst, MA, USA\\
  2015 - 2019  & Assistant Professor &  College of Information and Computer Sciences, University of Massachusetts Amherst, MA, USA\\
  2015 & Post-Doctoral Fellow &  Computer Science Department, Carnegie Mellon University, Pittsburgh, PA, USA\\
  2012 & Summer Intern &  Google Research, Mountain View, CA, USA\\
  2010 & Summer Intern &  Intel Research, Pittsburgh, PA, USA\\
\end{tabular}

%\pagebreak

\section*{Funding}

\subsection*{Federal Funding}

\textbf{NSF Award ``\emph{GCR: Community-Embedded Robotics: Understanding Sociotechnical Interactions with Long-term Autonomous Deployments
}''}\\
Role: Co-PI.\\
PI: Luis Sentis.\\
Co-PIs: Elliott Hauser, Justin Hart, Keri Stephens.\\
Period: October 2022 -- September 2027.\\
\funding{Amount: \$3,600,000}

\textbf{Army Research Laboratories Award ``\emph{Human-Guided Learning of Neuro-Symbolic Mission Execution Policies}''}\\
Role: PI.\\
Co-PI: Isil Dillig.\\
Period: September 2021 -- January 2023.\\
\funding{Amount: \$372,798}

\textbf{NSF Award ``\emph{NRT-AI: Convergent, Responsible, and Ethical Artificial Intelligence Training Experience for Roboticists}''}\\
Role: Co-PI.\\
PI: Junfeng Jiao.\\
Co-PIs: Luis Sentis, Justin Hart.\\
Period: September 2021 -- August 2026.\\
\funding{Amount: \$2,999,999.}

\textbf{NSF Award ``\emph{CAREER: Robust Perception and Customization for Long-Term Autonomous Mobile Service Robots}''}\\
Role: PI.\\
Period: April 2021 -- March 2026.\\
\funding{Amount: \$590,469.}

\textbf{NSF Award ``\emph{RI: Medium: Introspective Perception and Planning for Long-Term Autonomy}''}\\
Role: PI.\\
Co-PI: Shlomo Zilberstein (UMass)\\
Period: July 2020 -- June 2023.\\
\funding{PI Biswas' share: \$600,000.}

\textbf{NSF Award ``\emph{SHF: Small: Interactive Synthesis and Repair For Robot Programs}''}\\
Role: Co-PI.\\
PI: Arjun Guha (UMass)\\
Period: June 2020 -- May 2023.\\
\funding{PI Biswas' share: \$250,001.}

\textbf{DARPA Award ``\emph{Advancing Learning via Probabilistic Causal Analysis for Competency Awareness}''}\\
Role: Co-PI.\\
PI: Charles River Analytics
Co-PI: David Jensen (UMass).\\
Period: October 2019 -- September 2022.\\
\funding{PI Biswas' share: \$587,810.}

\textbf{Army Futures Command Robotics Center of Excellence ``\emph{Persistent Fully Autonomous Multi-Robot Tactics in Complex Environments}''}\\
Role: Co-PI.\\
PI: Peter Stone
Co-PIs: Luis Sentis, Justin Hart\\
Period: October 2019 -- December 2022.\\
\funding{Total funding: \$1,457,250.}

\textbf{NSF Award ``\emph{S\&AS: FND: Reliable Semi-Autonomy with Diminishing Reliance
on Humans}''}\\
Role: Co-PI.\\
PI: Shlomo Zilberstein (UMass)\\
Period: September 2017 -- August 2020.\\
\funding{Total funding: \$699,512.}

\textbf{DARPA Award ``\emph{Intelligent Model-Based Adaptation for Mobile
  Robotics}''}\\
Role: Co-PI.\\
PI: Jonathan Aldrich (CMU)
Co-PIs: David Garlan (CMU), Manuela Veloso
(CMU), Christian Kaestner (CMU), Claire Le Gouess (CMU).\\
Period: November 2015 -- November 2019.\\
\funding{PI Biswas' share: \$377,019.}

\vspace{-1.5em}
\subsection*{Competitive Industry Awards}

\textbf{JP Morgan Faculty Research Award, 2023}\\
Role: PI.\\
Collaborators: Arjun Guha (Northeastern University)\\
Period: September 2023 -- August 2024.\\
\funding{Amount: \$60,000.}

\textbf{Northrop Grumman Mission Systems’ Research in Applications for Learning Machines (REALM) Consortium}\\
Role: Co-PI.\\
PI: Shaoshuai Mou (Purdue)
Co-PIs: Daniel A. DeLaurentis (Purdue), Bing Liu (UIC)\\
Period: January 2019 -- December 2021.\\
\funding{Total funding: \$1,200,000. PI Biswas' share: \$302,668}

\textbf{JP Morgan AI Research Award, 2019}\\
Role: PI.\\
Period: September 2019 -- August 2020.\\
\funding{Amount: \$147,424.}

\textbf{Amazon Research Award, 2018}\\
Role: PI.\\
Period: September 2019 -- August 2020.\\
\funding{Amount: \$80,000.}

% \vspace{-1.5em}
% \subsection*{Industry Funding}

% \textbf{Northrop Grumman Unrestricted Gift}\\
% Award Date: February 2022\\
% \funding{Amount: \$75,000.}

% \textbf{Nissan Unrestricted Gift}\\*
% Award Date: February 2022\\*
% \funding{Amount: \$49,000.}

% \textbf{Amazon Unrestricted Gift}\\
% Award Date: December 2021\\
% \funding{Amount: \$70,000.}

% \textbf{Amazon Unrestricted Gift}\\
% Award Date: December 2020\\
% \funding{Amount: \$80,000.}

% \textbf{Amazon Unrestricted Gift}\\
% Award Date: March 2020\\
% \funding{Amount: \$80,000.}

% \textbf{Amazon Unrestricted Gift}\\
% Award Date: March 2019\\
% \funding{Amount: \$40,000.}

% \textbf{Amazon Unrestricted Gift}\\
% Award Date: September 2018\\
% \funding{Amount: \$40,000.}

% \clearpage
\section*{Teaching Experience}

\setlength{\parskip}{1em}
{\bf Instructor, CS 109, Fall 2023: The Essentials of AI for Life and Society}\\
University-wide course, University of Texas at Austin

{\bf Instructor, CS 388U, Fall 2023: Planning, Search, and Reasoning Under Uncertainty}\\
Online MS course, University of Texas at Austin

{\bf Instructor, CS 378H, Fall 2023: F1/10 Autonomous Driving -- Honors}\\
Honors Undergraduate course, University of Texas at Austin

{\bf Instructor, CS378/ME379M/ME397/ECE394J/ECE379K, Spring 2023: Connected  Autonomous Electric  Vehicles}\\
Undergraduate course, University of Texas at Austin

{\bf Instructor, CS 393R, Spring 2023: Planning, Search, and Reasoning Under Uncertainty}\\
Graduate course, University of Texas at Austin

{\bf Instructor, CS 378H, Spring 2022: F1/10 Autonomous Driving -- Honors}\\
Honors Undergraduate course, University of Texas at Austin

{\bf Instructor, CS 393R, Fall 2020, Fall 2021: Autonomous Robots}\\
Graduate course, University of Texas at Austin

{\bf Instructor, CS 378F, Spring 2020, Spring 2021: F1/10 Autonomous Driving}\\
Undergraduate course, University of Texas at Austin

{\bf Instructor, COMPSCI 220, Fall 2017, Fall 2018: Programming Methodology}\\
Undergraduate course, University of Massachusetts Amherst

{\bf Instructor, COMPSCI 403, Fall 2016, Spring 2018 : Introduction To
Robotics}\\
Undergraduate course, University of Massachusetts Amherst

{\bf Instructor, COMPSCI 603, Spring 2016, Spring 2017, Spring 2019 : Robotics}\\
\hfill Graduate course, University of Massachusetts Amherst

{\bf Instructor, COMPSCI 691BR, Spring 2017 : Building A Robot Soccer Team}\\
Graduate Seminar, University of Massachusetts Amherst

\begin{comment}
%%%%%%%%%%%%%%%%%%%%%%%%%%%%%%%%%%%%%%%%%%%%%%%%%%%%%%%%
Updates:
5. Find all SPC, PC appointments
6. List all students
%%%%%%%%%%%%%%%%%%%%%%%%%%%%%%%%%%%%%%%%%%%%%%%%%%%%%%%%
\end{comment}

\section*{Invited Talks}

Deploying Autonomous Service Mobile Robots, And Keeping Them Autonomous
\\
{\em University of Maryland}, October 2022

Deploying Autonomous Service Mobile Robots, And Keeping Them Autonomous
\\
{\em Northeastern University}, October 2022

Deploying Autonomous Service Mobile Robots, And Keeping Them Autonomous
\\
{\em Wellesley College}, October 2022

Deploying Autonomous Service Mobile Robots, And Keeping Them Autonomous
\\
{\em Samsung AI Center, NYC}, June 2022

Self-Supervised and User-Supervised Adaptation of Autonomous Robots
\\
{\em JP Morgan AI Research Center, NYC}, June 2022

Deploying Autonomous Service Mobile Robots, And Keeping Them Autonomous
\\
{\em Stanford University / Robotics Seminar}, April 2022

Deploying Autonomous Service Mobile Robots, And Keeping Them Autonomous
\\
{\em University of Southern California / CS Colloquium}, April 2022

Deploying Autonomous Service Mobile Robots, And Keeping Them Autonomous
\\
{\em Brown University / BigAI Talk}, April 2022

Motion Control and Visual Representation Learning for High-Speed Off-Road Driving
\\
{\em University of Pennsylvania / F1Tenth Invited Lecture}, April 2022

Particle Filters for Mobile Robot Localization
\\
{\em Wellesley College}, February 2022

Deploying Autonomous Service Mobile Robots, And Keeping Them Autonomous
\\
{\em Nvidia}, March 2021

Anticipating and Avoiding Failures Using Introspective Perception and Physics-Informed Program Synthesis\\
{\em MIT Embodied Intelligence Seminar}, February 2021

Building Robots For Long-Term Autonomy, And Keeping Them Autonomous\\
{\em Yale University}, April 2019

The Quest for "Always-On" Autonomous Mobile Robots\\
{\em IJCAI 2019 Early Career Spotlight Talk}, August 2019

Deploying Autonomous Service Mobile Robots, And Keeping Them Autonomous\\
{\em ICRA 2018 Workshop: Long-term Autonomy and Deployment of Intelligent Robots
in the Real-world}, May 2018

Building Robots For Long-Term Autonomy, And Keeping Them Autonomous\\
{\em Carnegie Mellon University},March 2018

Deploying Autonomous Service Mobile Robots, And Keeping Them Autonomous\\
{\em Amazon}, November 2017

Deploying Autonomous Service Mobile Robots, And Keeping Them Autonomous\\
{\em IROS 2017 Workshop: Assistance and Service Robotics in a Human
Environment}, September 2017

Autonomous Mobile Robot Perception for Changing Environments\\
{\em ICRA 2016 Workshop: AI for Long-term Autonomy}, May 2016

Deploying Autonomous Service Mobile Robots, And Keeping Them Autonomous.\\
{\em University of New Hampshire Robotics Seminar Series}, March 2016

The Quest for Robust, Reliable, Autonomous Mobile Robots.\\
{\em Williams College Computer Science Department Colloquium}, November 2015

The Quest for Robust, Reliable, Autonomous Mobile Robots.\\
{\em Vecna Robotics}, September 2015

The Quest for Robust, Reliable, Autonomous Mobile Robots.\\
{\em University of Minnesota, Computer Science \& Engineering}, April 2015

Vector Map-Based, Non-Markov Localization for Long-Term Deployment of Autonomous Mobile Robots\\
{\em Google X}, April 2015

The Quest for Robust, Reliable, Autonomous Mobile Robots.\\
{\em University of Massachusetts Amherst, School of Computer Science}, March 2015

The Quest for Robust, Reliable, Autonomous Mobile Robots.\\
{\em University of Massachusetts Amherst, School of Computer Science}, March 2015


\section*{Panels}
Panel Moderator, Ethics Aware Design of AI\\
{\em 2020 Global Analytics Summit: Ethics in AI, Texas McCombs}, November 2020

Discussion Panel, Record of Robotics at CMU Part II, A Live Interview with
Manuela Veloso.\\
{\em CMU Record of Robotics Series}, October 2020

Last Mile Autonomous Delivery Systems: A Live Webcast Demonstration, and Panel Discussion\\
{\em UT Good Systems Webinar}, September 2020

The Call for an Accelerated Autonomy -- Robotics on the Frontlines of a Crisis\\
{\em Computing In Our New Normal: A UTCS Webinar}, May 2020

Panel Chair, Reasoning and Learning in Real-World Systems for Long-Term
Autonomy\\
{\em AAAI 2018 Fall Symposium}, October 2018

\section*{Professional Service}

\subsection*{Outreach Activities}
\begin{itemize}
\item Science on Screen Series at Amherst Cinema, Amherst MA, 31 October 2018:
Presented a introduction to ``Christine'' within the scientific context of
actual self-driving cars. \emph{The Radical Future of Self-Driving Cars}.
\item SciTech Cafe, Northampton MA, 23 January 2017: Presented a scientific talk
to a general public audience. \emph{``Where am I?'' and
  Other Fundamental Questions Robotcs Think Long and Hard to Answer}
\item HolyokeCodes, Holyoke MA, 8--12 July 2019: Co-Organized with Arjun Guha, a week-long
robotics workshop for high-school students with state-of-the-art soccer-playing
robots that we used to compete with at RoboCup. We
covered the basic robot sense-plan-act control cycle, computational geometry,
and simple adversarial planning. Students implemented building blocks of
increasingly complex robot behaviors, leading up to a robot soccer tournament.
\end{itemize}

\subsection*{Track Chair / Associate Editor}

\begin{itemize}
  \item Co-Organizer, Texas Regional Robotics Symposium: April 29, 2022
  \item Associate Editor, Elsevier Robotics and Autonomous Systems: 2019 -- present
  \item RoboCup Federation Trustee: 2021 -- Present
  \item Diversity and Inclusion Co-Chair, AAAI Conference on Artificial Intelligence (AAAI): 2022
  \item RoboCup Executive Committee, Small Size League: 2015 -- 2021
  \item Robot Exhibitions Co-Chair, International Joint Conference on Artificial Intelligence (IJCAI): 2021
  \item RoboCup Symposium Co-Chair: 2020-2021
\item Robotics Track Co-Chair, International Conference on Autonomous Agents and
Multiagent Systems (AAMAS): 2019
  \item Associate Editor, IEEE/RSJ International Conference on Intelligent
Robots and Systems (IROS): 2016
\end{itemize}

\subsection*{Senior Program Committee}
\begin{itemize}
  \item International Joint Conference on Artificial Intelligence (IJCAI): 2022
  \item International Conference on Autonomous Agents and Multiagent Systems
  (AAMAS): 2022
  \item AAAI Conference on Artificial Intelligence (AAAI): 2020
  \item International Conference on Autonomous Agents and Multiagent Systems
  (AAMAS): 2021
  \item Autonomous Robots and Multirobot Systems Workshop (ARMS): 2016
\end{itemize}

\subsection*{Program Committee / Reviewer}
\begin{itemize}
 \item Autonomous Robots and Multirobot Systems Workshop (ARMS): 2020
 \item AAAI Symposium on Educational Advances in Artificial Intelligence: 2021
 \item RoboCup Symposium: 2015, 2016, 2017, 2018, 2019
 \item AAAI Undergraduate Consortium: 2021
 \item IEEE/SICE International Symposium on System Integration (SII): 2019
 \item Robotics: Science and Systems (RSS): 2015, 2016, 2019
 \item International Symposium on Multi-Robot and Multi-Agent Systems (MRS): 2019
 \item International Conference on Autonomous Agents and Multiagent Systems
(AAMAS): 2017, 2018
\item International Conference on Automated Planning and Scheduling (ICAPS)
 : 2016, 2018, 2021
 \item IEEE/RSJ International Conference on Intelligent Robots and Systems
(IROS): 2009, 2010, 2011, 2012, 2013, 2014, 2015, 2016, 2017, 2018, 2019, 2020
\item IEEE International Conference on Robotics and Automation (ICRA): 2010,
  2011, 2012, 2013, 2014, 2015, 2016, 2017, 2018, 2019, 2020, 2021
\item IEEE International Symposium on Robot and Human Interactive Communication
  (RO-MAN): 2010
 \item IEEE Conference on Human-Robot Interaction (HRI): 2016
 \item International Joint Conference on Artificial Intelligence (IJCAI): 2016
 \item International Conference on Autonomous Agents and Multiagent Systems
(AAMAS): 2016
\end{itemize}

\subsection*{Journal Reviewing}
\begin{itemize}
  \item IEEE Robotics and Automation Letters (RA-L): 2017, 2018, 2019, 2020, 2021
\item IEEE Robotics and Automation Magazine (IEEE-RAM): 2013, 2014, 2015, 2016, 2019
  \item IEEE Transactions on Robotics (T-RO): 2015, 2018, 2019, 2020
  \item International Journal of Robotics Research (IJRR): 2016, 2017
  \item International Journal of Social Robotics (SORO): 2018, 2019
\end{itemize}

\subsection*{Grant Reviewing}

NSF Panelist: 2016(x2), 2018, 2019, 2020, 2021, 2022


\section*{University Service}
\subsection*{College Level}
\begin{itemize}
  \item Faculty Hiring Committee for Whole Communities Whole Health Cluster
    Hires: 2019--2020
  \item CNS Fall Lab Working Group in response to COVID restrictions: 2020
\end{itemize}

\subsection*{Department Level}
\begin{itemize}
 \item Texas Robotics Machine Shop Committee: 2020--2022
 \item Texas Robotics Space Committee: 2020--2022
 \item UTCS Turing Scholars Admissions Committee: 2020--2022
 \item UTCS Diversity, Equity, and Inclusion Committee: 2020--2022
 \item UTCS Graduate Admissions Committee: 2019--2020
 \item UMass CICS Honors Program Director: 2018--2019
 \item UMass CICS Undergraduate Course Assistant Program Director: 2018--2019
 \item UMass CICS Graduate Admissions Committee: 2015--2016
 \item UMass CICS Student Activities Committee: 2015--2016
 \item UMass CICS Data Science Faculty Hiring Committee: 2016--2017
 \item UMass CICS Student Activities Committee: 2016--2017
\end{itemize}

\section*{Advising and Thesis Committees}

\subsection*{PostDoctoral Supervisor}
\begin{itemize}
 \item Rohan Chandra, 2022 -- present
 \item Kiarash Rahmani, 2022 -- present
\end{itemize}

\subsection*{PhD Supervisor}
\begin{itemize}
  \item Arthur Zhang, UT Austin. 2022--present
  \item Zichao Hu, UT Austin. 2022--present
  \item Sadanand Modak, UT Austin. 2022--present
  \item Noah Patton (Co-advised by Isil Dillig), UT Austin. 2022--present
  \item Eric Hsiung (Co-advised by Swarat Chaudhuri), UT Austin. 2022--present
 \item Amanda Adkins, UT Austin. 2020--present
 \item Joshua Hoffman (Co-advised by Swarat Chaudhuri), UT Austin. 2020--present
 \item Emily Pruc, UMass Amherst. 2018--2022
 \item Sadegh Rabiee, UT Austin. 2016--2022, currently at Amazon Lab126 \\
 \emph{Introspective Perception for Mobile Robots}
 \item Jarrett Holtz, UT Austin. 2015--2022, currently at Bosch Research \\
 \emph{Leveraging Program Synthesis for Robust Long-Term Robot Autonomy via\\ Interactive Learning and Adaptation}
 \item Samer Nashed (Changed advisors in 2019), UMass Amherst. 2015--2019
 \item Spencer Lane (Changed advisors in 2019), UMass Amherst. 2016--2019
 \item Alyxander Burns (Changed advisors in 2019), UMass Amherst. 2017--2019
\end{itemize}

\subsection*{Master's Thesis Supervisor}
\begin{itemize}
 \item Kavan Sikand, UT Austin. 2019--2022
 \item David Balaban, UMass Amherst, 2016 -- 2018\\
 \emph{A Real-Time Solver For Time-Optimal Control Of Omnidirectional Robots with Bounded Acceleration}
\end{itemize}

\subsection*{Undergraduate Honors Thesis Supervisor}
\begin{itemize}
 \item Elvin Yang, UT Austin. 2021--2022\\
 \emph{Wait, That Feels Familiar: Learning to Extrapolate Human Preferences for Preference-Aligned Path Planning}\\
 \textbf{Best Honors Thesis Award}
 \item Rahul Menon, UT Austin. 2021--2022\\
 \emph{Terrain-Adaptive Global Planning from Local Demonstrations}
 \item Shakeel Samsudeen, UT Austin. 2021--2022\\
 \emph{Context-Aware Object SLAM}
 \item Nathaniel Plaxton, UT Austin. 2021--2022\\
 \emph{Estimating Kinodynamic Uncertainty Using Learned Gaussian Noise Models}
 \item Michael Satanovski, UT Austin, 2021-2022\\
 \emph{An Empirical Evaluation of LIDAR Object Detectors for Autonomous Mobile Robots}
 \item Edward Schneeweiss, UMass Amherst, 2015 -- 2019\\
  \emph{Joint Perception and Planning for Obstacle Avoidance over Non-Planar Terrain}
 \item Kyle Vedder, UMass Amherst, 2015 -- 2019\\
  \emph{X*: Anytime Multiagent Path Planning With Bounded Search}
 \item George Larionov, UMass Amherst, 2015 -- 2016\\
  \emph{Human-robot Interaction: Integrating Speech Recognition with a Mobile Robot System}
\end{itemize}

\subsection*{PhD Committee Member}
\begin{itemize}
  \item Connor Basich, UMass Amherst. Supervisor: Shlomo Zilberstein
  \item Minkyu Kim, UT Austin. Supervisor: Luis Sentis
  \item Abhinav Verma, UT Austin. Supervisor: Swarat Chaudhuri
  \item Kyle Hollins Wray, UMass Amherst. Supervisor: Shlomo Zilberstein
  \item Justin Svegliato, UMass Amherst. Supervisor: Shlomo Zilberstein
  \item Tiffany Liu, UMass Amherst. Supervisor: Roderic Grupen
  \item Takeshi Takahashi, UMass Amherst. Supervisor: Roderic Grupen
  \item Mike Lanighan, UMass Amherst. Supervisor: Roderic Grupen
  \item Keen Sung, UMass Amherst. Supervisor: Brian Levine
  \item (Thesis Opponent)~\footnote{A PhD thesis dissertation in the Swedish
  doctoral system
is formally presented by an external examiner, called the \emph{thesis
opponent}. A thesis opponent places the work of the PhD thesis in context with
the state of the art, presents the findings of the thesis, and leads a
discussion with questions.}, Nils Bore, KTH. Supervisor: John Folkesson
\end{itemize}

\subsection*{Undergraduate Honors Thesis Committee Member}
\begin{itemize}
  \item Stefan Kussmaul, UMass Amherst. Supervisor: Roderic Grupen
  \item Karl Schmeckpepper, UMass Amherst. Supervisor: Roderic Grupen
\end{itemize}


\nocite{*}
\printbibheading
\bibbycategory


\end{document}


